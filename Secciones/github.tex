\section{Github} 
GitHub es una forja (plataforma de desarrollo colaborativo) para alojar proyectos utilizando el sistema de control de versiones Git. Se utiliza principalmente para la creación de código fuente de programas de computadora. El software que opera GitHub fue escrito en Ruby on Rails. Desde enero de 2010, GitHub opera bajo el nombre de GitHub, Inc. Anteriormente era conocida como Logical Awesome LLC. El código de los proyectos alojados en GitHub se almacena típicamente de forma pública, aunque utilizando una cuenta de pago, también permite hospedar repositorios privados.
\begin{itemize}
	\begin{center}
	\includegraphics[width=5cm]{./Imagenes/imagen1} 
	\end{center}
\end{itemize}


\section{¿Qué ofrece Github?} \\

Github ofrece al desarrollador toda la potencia y agilidad del sistema de control de versiones Git, más un interesante set de herramientas añadidas:
Wiki
Sistema de seguimiento de incidencias
Interfaz gráfica para revisión/comparación de código
Visor de ramas de desarrollo. \\


\section{¿Cómo funciona Github?} \\

Lo primero que debemos hacer es crear una cuenta en https://github.com. La activamos por mail y ya podemos crear nuestros repositorios. Los repositorios de Github son el almacén que utilizamos para guardar nuestro código. Github nos ofrece la opción de crear un repositorio vacío, recomendable cuando vamos a iniciar un nuevo desarrollo, o la opción de importar un proyecto ya existente, elegimos la que más nos convenga y mediante unos pocos comandos de consola configuramos la rama principal de nuestro repositorio, que por defecto se llamará "master". Cada programador puede crear sus propias ramas de desarrollo, donde tiene que llevar a cabo sus modificaciones, sin interferir en el trabajo de sus compañeros. Cuando terminamos y validamos un desarrollo paralelo, lo unimos con la rama principal y todos los miembros del equipo pueden descargar las nuevas modificaciones, sin alterar los desarrollos que estén llevando cabo en ese momento. Después de alojar el repositorio público en Github.com, cualquier usuario de la comunidad podrá aportar ideas, hacer un seguimiento del proyecto, incluso copiarlo y modificarlo a su gusto bajo la misma licencia.

